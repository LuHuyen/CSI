\documentclass[a4paper,12pt]{report}

\usepackage[french]{babel}

\usepackage[a4paper,top=2cm,bottom=2cm,left=3cm,right=3cm,marginparwidth=1.75cm]{geometry}

% Useful packages
\usepackage{amsmath, amsfonts, amssymb}
\usepackage{graphicx}
\usepackage{color}
\usepackage[colorlinks=true, allcolors=blue]{hyperref}
\usepackage{tikz}
\usepackage{pgfplots}
\usepackage{tikz}
\usetikzlibrary{3d,calc}
\pgfplotsset{compat=1.18}
\usepackage[utf8]{inputenc}
\usepackage{cleveref}

\title{Rapport d'activités}
\author{Viet Anh QUACH}

\begin{document}
\maketitle
\tableofcontents

\setcounter{secnumdepth}{4}

\chapter{Introduction de la thèse}
% 
À cause du changement climatique, les risques naturels concernant l'écoulement gravitaire sur les zones montagneuses sont déclenchés de plus en plus fréquent à notre époque. 
Récemmement, Le village de Blatten, dans les Alpes suisses est rayé de la carte sous l'effet de L’éboulement du glacier le 29 mai 2025. 
Malhereusement, la caractéristique mécanique de ces écoulements gravitaires, reste mal comprise. 
Donc, mon projet de thèse, " Homogénéisation numérique à double échelle pour la modélisation demouvements gravitaires liés aux changements climatiques", se porte sur le développement d'une modélisation mécanique numérique pour l’écoulement de matériaux complexes. 
La méthodologie numérique pour réaliser ce simulation est basé sur La modélisation en intégrer 2 échelle simultané, qui consiste 2 processus hiérarchiques de each other. 
Pour cette homogénéisation, LA Méthode lagrangienne de Points Matériels (MPM) est choisisé à gestionner  La macroscopic, handle la mobilité du l'écoulement. 
meanwhile la microscopic, assuré par Méthode des éléments discrètes (DEM) jou la role de Loi Constitutive Homogénéisée Numériquement (LCHN).
Effectivement, La couplage entre 2 méthodes différents has been used prevelantly est efficacely, sans peut pas ignorer comme FEm x DEM \cite{nguyen2013modelisation}. Mais pour résourdre ce probleme du écoulement granulaire, les 2 adopté méthodes bright shinely et apporte des avantages significatifs.
For instance, comme indiqué dans \autoref{fig:MPMxDEM}, la MPM prends en charge de la computation du gradient de déformation, envoyant au DEM à calculer la tensor du contrainte selon la formulation de LOVE,WEBER et la cycle continue. 
Grace à l'acces du DEM, la résolution intergranulaire important du petit échelle est accesiblé et manipulé liberté afin de s'adapter au complexité du matériaux modelisé. 
One peut changer la taille, shape, rigidité, ou bien quelle indice qui s'adapté au matériau d'étude le plus précis. 
En meme temps, MPM reduit la complexité du prix chère du calcul en le considérant comme un object total mais sans facer à la subis de grand déformation qui distorte la maillage comme FEM.

\vspace{2em}
Au laboratoire 3SR, deux programmes de calcul en C++ pour la MPM et la DEM s'appellent respectivement ``MPMbox'' et ``3D\_sandstone'' et leur intégration est bien implémentée~\cite{richefeu2025mpmxdem}.
La modélisation est aussi réalisée et prouvée précise, mais il reste des verrous à surmonter:
\begin{itemize}
    \item Équivalence des inerties aux deux échelles : Indeed, MPM qui a été developpé pour handle des problemes contact des objets \cite{nguyen2023material.}, porte bien l'effet dynamique. En revanche, DEM censure les réponses mécaniques dynamiques en attendant une stabilisation statique systématique \cite{nguyen2014fem}. Leur couplage fait emerge des Diverses nouvelles interrogations théoriques sur l'équivalence des inerties aux deux échelles together en meme temps, when it comes to computational calcul for a granular flow which is highly dynamic.
    \item Conditions aux limites et initiales~: spécifiquement, dans la simulation DEM, la partie de condition aux limites périodiques n'est pas réalisée de manière pleinement satisfaisante, ce qui pose des pics anormaux de contrainte.
\end{itemize}
Le détail est bien illustré dans les sections suivante



\chapter{Étude bibliographique}
\section{MPM}
\begin{itemize}
\item Thèse: Application de la Méthode des Points Matériels aux phénomènes gravitaires - Fabio GRACIA DANIES
\item A multi-scale, MPMxDEM, numerical modelling approach for geotechnical structures under severe loading - Sacha Duverger
\item Material Point Method -- Vinh Phu NGUYEN, Alban de Vaucorbeil, Stephane Bordas
\end{itemize}
\section{DEM}
\begin{itemize}
      \item \cite{combe2023demlecture}
      \item Thèse: Modélisation multi-échelle des matériaux granulaires frottant-cohésifs: Trung Kien NGUYEN
\end{itemize}

\section{Couplage MPMxDEM}
\begin{itemize}
      \item Thèse: Application de la Méthode des Points Matériels aux phénomènes gravitaires - Fabio GRACIA DANIES
      \item \cite{richefeu2025mpmxdem}
\end{itemize}
\begin{figure}
\centering
\includegraphics[width=0.3\textwidth]{CouplageMPMxDEM.png}
\caption{Principe d’une approche à deux échelles hiérarchiques.\cite{projetderecherche}}
\label{fig:MPMxDEM}
\end{figure}
La corrélation entre les deux échelles est décrite dans la \autoref{fig:MPMxDEM}. 
% La partie à gauche représente un système MPM où on peut voir les
% nœuds (bleus) de grille fixe, les points matériels (rouges). La partie à droite montre un volume
% périodique DEM se déformant sous la demande d’un gradient de vitesse de la part d’un point
% matériel, et répondant par une contrainte obtenue par homogénéisation numérique.

\section{Compétences soutenues}
\begin{itemize}
      \item C++
      \item Gnuplot
      \item LaTeX
\end{itemize}

% Inkscape
% Writing thesis
% 120 formations sur ADUM
% Linux
% Clavier français :D

\chapter{L'étude pratique}
\section{Simulation MPM}
\subsection{Étude sur le cas statique - Déformation d'une poutre console}
\subsubsection{PIC}
\subsubsection{coefficient Poisson}
\subsubsection{Longueur encastrée}
\subsubsection{Discrétisation du maillage}
\subsubsection{Comparaison entre calcul théorique et numérique}
                                \begin{figure}
                                   \centering\small\input{Deplacement_Poutre.tex}
                                    \caption{Déplacement du poutre en Y-axis}
                                \end{figure}
\subsection{Étude sur le cas dynamique}


\section{Simulation DEM: Compression triaxiale}
\subsection{Le processus et les paramètres}
\subsection{Les caractéristiques mécaniques générales du sol}
\subsubsection{La granulométrie et la fraction solide}
% \begin{figure}[!h]
%   \centering
%   \subfloat[\texttt{Illustrator} and \texttt{LaTeXiT}]{\includegraphics{Taylor-bar-setup}\label{fig:figures11a}}
%   \subfloat[\texttt{Inkscape} and PDF\_TEX]{\input{Taylor-bar-setup_pdf_tex.pdf_tex}\label{fig:figures11b}}
%   \caption{Using \texttt{Illustrator} and \texttt{Inskcape} to produce vector images with \LaTeX\ symbols. The font in figure (a) is slightly different from the one in the text ($\nabla\cdot\boldsymbol{\sigma} = \boldsymbol{\mathit{0}}$), while it matches perfectly in figure (b).}
%   \label{fig:figures11}
% \end{figure}
\subsubsection{La granulométrie et la fraction solide}
La fraction solide caractère la dispersion des particules solide dans une volumne.
\[
\Phi = \dfrac{V_s}{V} = \dfrac{\sum\limits_{i = 1}^n \frac{4}{3}\pi R_i^3}{\det(h)}
\]
                                \begin{figure}
                                   \centering \small \input{fractonSolide.tex}
                                    \caption{La fraction solide}
                                    \label{fig:fractionSolideMax}
                                \end{figure}

La \autoref{fig:fractionSolideMax} montre que leur valeur maximale $\Phi_{\max}$ pour une dispersion désordonnée de sphères en contact dense est approximée à 0.64~\cite{combe2023demlecture}.
\subsubsection{Les caractéristiques des échantillons denses}
\subsubsection{Les caractéristiques des échantillons lâches}
\subsubsection{L'état critique}
\paragraph{La force entre les grains}
\paragraph{L'indice de vide}
\paragraph{Le nombre de coordination}
\subsubsection{Cercle de Mohr}
\subsubsection{Histoire du chargement}

\subsection{Recherche sur l'impact dynamique}

\subsubsection{Augmentation du nombre d'inertie (Montée la vitesse imposée)}


\section{Couplage\ldots}
\subsection{How to add Tables}

Use the table and tabular environments for basic tables --- see Table~\ref{tab:widgets}, for example. For more information, please see this help article on \href{https://www.overleaf.com/learn/latex/tables}{tables}. 

\begin{table}
\centering
\begin{tabular}{l|r}
Item & Quantity \\\hline
Widgets & 42 \\
Gadgets & 13
\end{tabular}
\caption{\label{tab:widgets}An example table.}
\end{table}

\subsection{How to add Lists}

You can make lists with automatic numbering \dots

\begin{enumerate}
\item Like this,
\item and like this.
\end{enumerate}
\dots or bullet points \dots
\begin{itemize}
\item Like this,
\item and like this.
\end{itemize}
\chapter{Conclusion}

                                \begin{figure}
                                    \centering
                                  \small
                                  \input{RapportVides.tex}
                                    \caption{L'indice de vide}
                                \end{figure}
                                    
                                \begin{figure}
                                   % GNUPLOT: LaTeX picture with Postscript
\begingroup
  \makeatletter
  \providecommand\color[2][]{%
    \GenericError{(gnuplot) \space\space\space\@spaces}{%
      Package color not loaded in conjunction with
      terminal option `colourtext'%
    }{See the gnuplot documentation for explanation.%
    }{Either use 'blacktext' in gnuplot or load the package
      color.sty in LaTeX.}%
    \renewcommand\color[2][]{}%
  }%
  \providecommand\includegraphics[2][]{%
    \GenericError{(gnuplot) \space\space\space\@spaces}{%
      Package graphicx or graphics not loaded%
    }{See the gnuplot documentation for explanation.%
    }{The gnuplot epslatex terminal needs graphicx.sty or graphics.sty.}%
    \renewcommand\includegraphics[2][]{}%
  }%
  \providecommand\rotatebox[2]{#2}%
  \@ifundefined{ifGPcolor}{%
    \newif\ifGPcolor
    \GPcolortrue
  }{}%
  \@ifundefined{ifGPblacktext}{%
    \newif\ifGPblacktext
    \GPblacktextfalse
  }{}%
  % define a \g@addto@macro without @ in the name:
  \let\gplgaddtomacro\g@addto@macro
  % define empty templates for all commands taking text:
  \gdef\gplbacktext{}%
  \gdef\gplfronttext{}%
  \makeatother
  \ifGPblacktext
    % no textcolor at all
    \def\colorrgb#1{}%
    \def\colorgray#1{}%
  \else
    % gray or color?
    \ifGPcolor
      \def\colorrgb#1{\color[rgb]{#1}}%
      \def\colorgray#1{\color[gray]{#1}}%
      \expandafter\def\csname LTw\endcsname{\color{white}}%
      \expandafter\def\csname LTb\endcsname{\color{black}}%
      \expandafter\def\csname LTa\endcsname{\color{black}}%
      \expandafter\def\csname LT0\endcsname{\color[rgb]{1,0,0}}%
      \expandafter\def\csname LT1\endcsname{\color[rgb]{0,1,0}}%
      \expandafter\def\csname LT2\endcsname{\color[rgb]{0,0,1}}%
      \expandafter\def\csname LT3\endcsname{\color[rgb]{1,0,1}}%
      \expandafter\def\csname LT4\endcsname{\color[rgb]{0,1,1}}%
      \expandafter\def\csname LT5\endcsname{\color[rgb]{1,1,0}}%
      \expandafter\def\csname LT6\endcsname{\color[rgb]{0,0,0}}%
      \expandafter\def\csname LT7\endcsname{\color[rgb]{1,0.3,0}}%
      \expandafter\def\csname LT8\endcsname{\color[rgb]{0.5,0.5,0.5}}%
    \else
      % gray
      \def\colorrgb#1{\color{black}}%
      \def\colorgray#1{\color[gray]{#1}}%
      \expandafter\def\csname LTw\endcsname{\color{white}}%
      \expandafter\def\csname LTb\endcsname{\color{black}}%
      \expandafter\def\csname LTa\endcsname{\color{black}}%
      \expandafter\def\csname LT0\endcsname{\color{black}}%
      \expandafter\def\csname LT1\endcsname{\color{black}}%
      \expandafter\def\csname LT2\endcsname{\color{black}}%
      \expandafter\def\csname LT3\endcsname{\color{black}}%
      \expandafter\def\csname LT4\endcsname{\color{black}}%
      \expandafter\def\csname LT5\endcsname{\color{black}}%
      \expandafter\def\csname LT6\endcsname{\color{black}}%
      \expandafter\def\csname LT7\endcsname{\color{black}}%
      \expandafter\def\csname LT8\endcsname{\color{black}}%
    \fi
  \fi
    \setlength{\unitlength}{0.0500bp}%
    \ifx\gptboxheight\undefined%
      \newlength{\gptboxheight}%
      \newlength{\gptboxwidth}%
      \newsavebox{\gptboxtext}%
    \fi%
    \setlength{\fboxrule}{0.5pt}%
    \setlength{\fboxsep}{1pt}%
    \definecolor{tbcol}{rgb}{1,1,1}%
\begin{picture}(7200.00,5040.00)%
    \gplgaddtomacro\gplbacktext{%
      \csname LTb\endcsname%%
      \put(814,1604){\makebox(0,0)[r]{\strut{}$1$}}%
      \csname LTb\endcsname%%
      \put(814,1961){\makebox(0,0)[r]{\strut{}$1.5$}}%
      \csname LTb\endcsname%%
      \put(814,2318){\makebox(0,0)[r]{\strut{}$2$}}%
      \csname LTb\endcsname%%
      \put(814,2676){\makebox(0,0)[r]{\strut{}$2.5$}}%
      \csname LTb\endcsname%%
      \put(814,3033){\makebox(0,0)[r]{\strut{}$3$}}%
      \csname LTb\endcsname%%
      \put(814,3390){\makebox(0,0)[r]{\strut{}$3.5$}}%
      \csname LTb\endcsname%%
      \put(814,3747){\makebox(0,0)[r]{\strut{}$4$}}%
      \csname LTb\endcsname%%
      \put(814,4105){\makebox(0,0)[r]{\strut{}$4.5$}}%
      \csname LTb\endcsname%%
      \put(814,4462){\makebox(0,0)[r]{\strut{}$5$}}%
      \csname LTb\endcsname%%
      \put(814,4819){\makebox(0,0)[r]{\strut{}$5.5$}}%
      \csname LTb\endcsname%%
      \put(946,1384){\makebox(0,0){\strut{}$0$}}%
      \csname LTb\endcsname%%
      \put(2117,1384){\makebox(0,0){\strut{}$20$}}%
      \csname LTb\endcsname%%
      \put(3289,1384){\makebox(0,0){\strut{}$40$}}%
      \csname LTb\endcsname%%
      \put(4460,1384){\makebox(0,0){\strut{}$60$}}%
      \csname LTb\endcsname%%
      \put(5632,1384){\makebox(0,0){\strut{}$80$}}%
      \csname LTb\endcsname%%
      \put(6803,1384){\makebox(0,0){\strut{}$100$}}%
    }%
    \gplgaddtomacro\gplfronttext{%
      \csname LTb\endcsname%%
      \put(209,3211){\rotatebox{-270}{\makebox(0,0){\strut{}Z}}}%
      \put(3874,1054){\makebox(0,0){\strut{}$\varepsilon_{yy}$ (\%)}}%
      \csname LTb\endcsname%%
      \put(1980,813){\makebox(0,0)[r]{\strut{}$I = 1 \times 10^{-4}$}}%
      \csname LTb\endcsname%%
      \put(1980,513){\makebox(0,0)[r]{\strut{}$I = 1 \times 10^{-3}$}}%
      \csname LTb\endcsname%%
      \put(1980,213){\makebox(0,0)[r]{\strut{}$I = 2 \times 10^{-3}$}}%
      \csname LTb\endcsname%%
      \put(4203,813){\makebox(0,0)[r]{\strut{}$I = 4 \times 10^{-3}$}}%
      \csname LTb\endcsname%%
      \put(4203,513){\makebox(0,0)[r]{\strut{}$I = 6 \times 10^{-3}$}}%
      \csname LTb\endcsname%%
      \put(4203,213){\makebox(0,0)[r]{\strut{}$I = 8 \times 10^{-3}$}}%
      \csname LTb\endcsname%%
      \put(6426,813){\makebox(0,0)[r]{\strut{}$I = 1 \times 10^{-2}$}}%
    }%
    \gplbacktext
    \put(0,0){\includegraphics[width={360.00bp},height={252.00bp}]{./NombreCoordination}}%
    \gplfronttext
  \end{picture}%
\endgroup

                                    \caption{Nombre de coordination}
                                \end{figure}

                                Vitesse imposée est élevée $\rightarrow$ Contrainte de confinement et l'indice de vide sont instables

     
                                \begin{figure}
                                    \input{Test.tex}
                                    \caption{Contrainte-Déformation}
                                \end{figure}
                                J'ai choisi $\varepsilon_{yy} = 70\%$ pour réaliser des pré-études sur la rhéologie $\mu(I)$ (considéré comme l'état critique). 


                                \begin{figure}
                                  \input{Transitoire.tex}
                                    \caption{Cercle transitore (pic)}
                                \end{figure}

                                    $\varphi = 28.77^\circ \div 39.53^\circ $


                                \begin{figure}
                                    % GNUPLOT: LaTeX picture with Postscript
\begingroup
  \makeatletter
  \providecommand\color[2][]{%
    \GenericError{(gnuplot) \space\space\space\@spaces}{%
      Package color not loaded in conjunction with
      terminal option `colourtext'%
    }{See the gnuplot documentation for explanation.%
    }{Either use 'blacktext' in gnuplot or load the package
      color.sty in LaTeX.}%
    \renewcommand\color[2][]{}%
  }%
  \providecommand\includegraphics[2][]{%
    \GenericError{(gnuplot) \space\space\space\@spaces}{%
      Package graphicx or graphics not loaded%
    }{See the gnuplot documentation for explanation.%
    }{The gnuplot epslatex terminal needs graphicx.sty or graphics.sty.}%
    \renewcommand\includegraphics[2][]{}%
  }%
  \providecommand\rotatebox[2]{#2}%
  \@ifundefined{ifGPcolor}{%
    \newif\ifGPcolor
    \GPcolortrue
  }{}%
  \@ifundefined{ifGPblacktext}{%
    \newif\ifGPblacktext
    \GPblacktextfalse
  }{}%
  % define a \g@addto@macro without @ in the name:
  \let\gplgaddtomacro\g@addto@macro
  % define empty templates for all commands taking text:
  \gdef\gplbacktext{}%
  \gdef\gplfronttext{}%
  \makeatother
  \ifGPblacktext
    % no textcolor at all
    \def\colorrgb#1{}%
    \def\colorgray#1{}%
  \else
    % gray or color?
    \ifGPcolor
      \def\colorrgb#1{\color[rgb]{#1}}%
      \def\colorgray#1{\color[gray]{#1}}%
      \expandafter\def\csname LTw\endcsname{\color{white}}%
      \expandafter\def\csname LTb\endcsname{\color{black}}%
      \expandafter\def\csname LTa\endcsname{\color{black}}%
      \expandafter\def\csname LT0\endcsname{\color[rgb]{1,0,0}}%
      \expandafter\def\csname LT1\endcsname{\color[rgb]{0,1,0}}%
      \expandafter\def\csname LT2\endcsname{\color[rgb]{0,0,1}}%
      \expandafter\def\csname LT3\endcsname{\color[rgb]{1,0,1}}%
      \expandafter\def\csname LT4\endcsname{\color[rgb]{0,1,1}}%
      \expandafter\def\csname LT5\endcsname{\color[rgb]{1,1,0}}%
      \expandafter\def\csname LT6\endcsname{\color[rgb]{0,0,0}}%
      \expandafter\def\csname LT7\endcsname{\color[rgb]{1,0.3,0}}%
      \expandafter\def\csname LT8\endcsname{\color[rgb]{0.5,0.5,0.5}}%
    \else
      % gray
      \def\colorrgb#1{\color{black}}%
      \def\colorgray#1{\color[gray]{#1}}%
      \expandafter\def\csname LTw\endcsname{\color{white}}%
      \expandafter\def\csname LTb\endcsname{\color{black}}%
      \expandafter\def\csname LTa\endcsname{\color{black}}%
      \expandafter\def\csname LT0\endcsname{\color{black}}%
      \expandafter\def\csname LT1\endcsname{\color{black}}%
      \expandafter\def\csname LT2\endcsname{\color{black}}%
      \expandafter\def\csname LT3\endcsname{\color{black}}%
      \expandafter\def\csname LT4\endcsname{\color{black}}%
      \expandafter\def\csname LT5\endcsname{\color{black}}%
      \expandafter\def\csname LT6\endcsname{\color{black}}%
      \expandafter\def\csname LT7\endcsname{\color{black}}%
      \expandafter\def\csname LT8\endcsname{\color{black}}%
    \fi
  \fi
    \setlength{\unitlength}{0.0500bp}%
    \ifx\gptboxheight\undefined%
      \newlength{\gptboxheight}%
      \newlength{\gptboxwidth}%
      \newsavebox{\gptboxtext}%
    \fi%
    \setlength{\fboxrule}{0.5pt}%
    \setlength{\fboxsep}{1pt}%
    \definecolor{tbcol}{rgb}{1,1,1}%
\begin{picture}(7200.00,5040.00)%
    \gplgaddtomacro\gplbacktext{%
      \csname LTb\endcsname%%
      \put(814,2195){\makebox(0,0)[r]{\strut{}$0$}}%
      \put(814,2613){\makebox(0,0)[r]{\strut{}$50$}}%
      \put(814,3032){\makebox(0,0)[r]{\strut{}$100$}}%
      \put(814,3450){\makebox(0,0)[r]{\strut{}$150$}}%
      \put(814,3868){\makebox(0,0)[r]{\strut{}$200$}}%
      \put(946,1975){\makebox(0,0){\strut{}$0$}}%
      \put(1783,1975){\makebox(0,0){\strut{}$100$}}%
      \put(2619,1975){\makebox(0,0){\strut{}$200$}}%
      \put(3456,1975){\makebox(0,0){\strut{}$300$}}%
      \put(4293,1975){\makebox(0,0){\strut{}$400$}}%
      \put(5130,1975){\makebox(0,0){\strut{}$500$}}%
      \put(5966,1975){\makebox(0,0){\strut{}$600$}}%
      \put(6803,1975){\makebox(0,0){\strut{}$700$}}%
    }%
    \gplgaddtomacro\gplfronttext{%
      \csname LTb\endcsname%%
      \put(209,3031){\rotatebox{-270}{\makebox(0,0){\strut{}$\tau$ (kPa)}}}%
      \put(3874,1645){\makebox(0,0){\strut{}$\sigma_n$ (kPa)}}%
      \csname LTb\endcsname%%
      \put(2160,513){\makebox(0,0)[r]{\strut{}$I = 1 \times 10^{-4}$}}%
      \csname LTb\endcsname%%
      \put(2160,333){\makebox(0,0)[r]{\strut{}$I = 1 \times 10^{-3}$}}%
      \csname LTb\endcsname%%
      \put(2160,153){\makebox(0,0)[r]{\strut{}$I = 2 \times 10^{-3}$}}%
      \csname LTb\endcsname%%
      \put(4167,513){\makebox(0,0)[r]{\strut{}$I = 4 \times 10^{-3}$}}%
      \csname LTb\endcsname%%
      \put(4167,333){\makebox(0,0)[r]{\strut{}$I = 6 \times 10^{-3}$}}%
      \csname LTb\endcsname%%
      \put(4167,153){\makebox(0,0)[r]{\strut{}$I = 8 \times 10^{-3}$}}%
      \csname LTb\endcsname%%
      \put(6174,513){\makebox(0,0)[r]{\strut{}$I = 1 \times 10^{-2}$}}%
    }%
    \gplbacktext
    \put(0,0){\includegraphics[width={360.00bp},height={252.00bp}]{Résiduel}}%
    \gplfronttext
  \end{picture}%
\endgroup

                                    \caption{Cercle résiduel ($\varepsilon_{yy} = 70\%$)}
                                \end{figure}
                                    $\varphi = 16.56^\circ \div 22.78^\circ $
                                    
\section{Comparaison des équations utilisées}

Expérimental : 
\[
I =  \dfrac{\dot{\varepsilon} \cdot \bar{a}}{\sqrt{\sigma_{33}/\rho_s}}; \quad \mu = \sin(\varphi)
\]

Simulation : 
\[
I = \dot{\varepsilon} \times \sqrt {\dfrac{m}{\sigma_{33}\times \bar{a}}}  
= \dot{\varepsilon} \times \sqrt {\dfrac{\dfrac{4}{3} \pi \dfrac{\bar{a}^3}{8} \times \rho_s}{\sigma_{33}\times \bar{a}}} 
= \dot{\varepsilon} \times \sqrt{\dfrac{\pi}{6}} \sqrt {\dfrac{\bar{a}^2 \rho_s}{\sigma_{33}}} 
= \boxed{\sqrt{\dfrac{\pi}{6}}} \times \dfrac{\dot{\varepsilon} \cdot \bar{a}}{\sqrt{\sigma_{33}/\rho_s}}
\]
\[
\mu = \tan(\varphi)
\]

\section{Rhéologie $\mu(I)$ résiduel}

\subsection{Première méthode}

La première méthode utilise les équations suivantes :
\[
\mu(I) = \mu_s + \dfrac{\mu_2 - \mu_s}{1 + \dfrac{I_0}{I}}
\]
\[
\Phi(I) = \Phi^{\max} - bI
\]

Les coefficients $\mu_s,\ \mu_2,\ I_0,\ \Phi_{\max},\ b$ sont déterminés empiriquement.

\begin{figure}
    \centering
    {\small
        \input{mu_I_fit.tex}
    }
    \caption{$\mu(I)$ ($\varepsilon_{yy} = 70\%$)}
\end{figure}

\begin{figure}
    \centering
    {\small
        \input{Pack_I_fit.tex}
    }
    \caption{$\Phi(I)$ ($\varepsilon_{yy} = 70\%$)}
\end{figure}

\subsection{Deuxième méthode}

La deuxième méthode, proposée dans l'article 
\href{https://link-springer-com.sid2nomade-1.grenet.fr/article/10.1007/s10035-024-01459-7}{Scaling laws for quasi-statically deforming granular soil at critical state [2024] (Fei, Jianbo et al.)}, 
introduit un nombre d'inertie quasi-statique $Q$ qui tient compte du degré de compaction $\Phi_0$ :

\[
Q = \left[ \Phi_0  \ln \left( I \right) + \alpha \right]
\]
où $\alpha = 30$.

\[
\mu = \xi Q + C
\]

Les coefficients $\xi,\ C,\ \Phi_0$ sont déterminés empiriquement ($\xi \approx 0.06$ et $C \approx 0.2$).

\begin{figure}
    \centering
    {\small
        \input{Q_mu.tex}
    }
    \caption{$\mu(Q)$ avec $Q = f(\Phi_0, I)$}
\end{figure}

\bibliographystyle{plain}
\bibliography{Bibliographie}


\end{document}


