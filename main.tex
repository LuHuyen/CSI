\documentclass[a4paper,12pt]{report}

\usepackage[french]{babel}

\usepackage[a4paper,top=2cm,bottom=2cm,left=3cm,right=3cm,marginparwidth=1.75cm]{geometry}

% Useful packages
\usepackage{amsmath, amsfonts, amssymb}
\usepackage{graphicx}
\usepackage[colorlinks=true, allcolors=blue]{hyperref}
\usepackage{tikz}
\usepackage{pgfplots}
\pgfplotsset{compat=1.18}

\title{Rapport d'activités}
\author{Viet Anh QUACH}

\begin{document}
\maketitle
\tableofcontents

\setcounter{secnumdepth}{4}

\chapter{Introduction de la thèse}

\chapter{Bibliographie}
\section{MPM}
\begin{itemize}
\item Thèse: Application de la Méthode des Points Matériels aux phénomènes gravitaires - Fabio GRACIA DANIES
\item A multi-scale, MPMxDEM, numerical modelling approach for geotechnical structures under severe loading - Sacha Duverger
\item Material Point Method -- Vinh Phu NGUYEN, Alban de Vaucorbeil, Stephane Bordas
\end{itemize}
\section{DEM}
\begin{itemize}
      \item DEM-Lecture-Version2023 - Gaël COMBE
      \item Thèse: Modélisation multi-échelle des matériaux granulaires frottant-cohésifs: Trung Kien NGUYEN
\end{itemize}

\section{Couplage MPMxDEM}
\begin{itemize}
      \item Thèse: Application de la Méthode des Points Matériels aux phénomènes gravitaires - Fabio GRACIA DANIES
      \item V. Richefeu, G. Combe, MPM×DEM programme, https://github.com/richefeu/mpmxdem (2025)
\end{itemize}
\section{Corrélation}

\section{Compétences soutenues}
\begin{itemize}
      \item C++
      \item Gnuplot
      \item LaTeX
\end{itemize}

% Inkscape
% Writing thesis
% 120 formations sur ADUM
% Linux
% Clavier français :D

\chapter{L'étude pratique}
\section{Simulation MPM}
\subsection{Étude sur le cas statique - Déformation d'une poutre console}
\subsubsection{PIC}
\subsubsection{coefficient Poisson}
\subsubsection{Longueur encastrée}
\subsubsection{Discrétisation du maillage}
\subsubsection{Comparaison entre calcul théorique et numérique}
                                \begin{figure}
                                    \centering
                                    \scalebox{0.5}{\input{}}
                                    \caption{Cercle transitore (pic)}
                                \end{figure}
\subsection{Étude sur le cas dynamique}


\section{Simulation DEM: Compression triaxiale}
\subsection{Le processus et les paramètres}
\subsection{Les caractéristiques mécaniques générales du sol}
\subsubsection{La granulométrie et la fraction solide}
\subsubsection{La granulométrie et la fraction solide}
\subsubsection{Les caractéristiques des échantillons denses}
\subsubsection{Les caractéristiques des échantillons lâches}
\subsubsection{L'état critique}
\paragraph{La force entre les grains}
\paragraph{L'indice de vide}
\paragraph{Le nombre de coordination}
\subsubsection{Cercle de Mohr}
\subsubsection{Histoire du chargement}

\subsection{Recherche sur l'impact dynamique}

\subsubsection{Augmentation du nombre d'inertie (Montée la vitesse imposée)}
\begin{figure}
\centering
\includegraphics[width=0.3\textwidth]{frog.jpg}
\caption{\label{fig:frog}This frog was uploaded via the file-tree menu.}
\end{figure}

\section{Couplage\ldots}
\subsection{How to add Tables}

Use the table and tabular environments for basic tables --- see Table~\ref{tab:widgets}, for example. For more information, please see this help article on \href{https://www.overleaf.com/learn/latex/tables}{tables}. 

\begin{table}
\centering
\begin{tabular}{l|r}
Item & Quantity \\\hline
Widgets & 42 \\
Gadgets & 13
\end{tabular}
\caption{\label{tab:widgets}An example table.}
\end{table}

\subsection{How to add Lists}

You can make lists with automatic numbering \dots

\begin{enumerate}
\item Like this,
\item and like this.
\end{enumerate}
\dots or bullet points \dots
\begin{itemize}
\item Like this,
\item and like this.
\end{itemize}
\chapter{Conclusion}

\section{Comparaison des équations utilisées}

Expérimental : 
\[
I =  \frac{\dot{\varepsilon} \cdot \bar{a}}{\sqrt{\sigma_{33}/\rho_s}}; \quad \mu = \sin(\varphi)
\]

Simulation : 
\[
I = \dot{\varepsilon} \times \sqrt {\frac{m}{\sigma_{33}\times \bar{a}}}  
= \dot{\varepsilon} \times \sqrt {\frac{\frac{4}{3} \pi \frac{\bar{a}^3}{8} \times \rho_s}{\sigma_{33}\times \bar{a}}} 
= \dot{\varepsilon} \times \sqrt{\frac{\pi}{6}} \sqrt {\frac{\bar{a}^2 \rho_s}{\sigma_{33}}} 
= \boxed{\sqrt{\frac{\pi}{6}}} \times \frac{\dot{\varepsilon} \cdot \bar{a}}{\sqrt{\sigma_{33}/\rho_s}}
\]
\[
\mu = \tan(\varphi)
\]

\section{Rhéologie $\mu(I)$ résiduel}

\subsection{Première méthode}

La première méthode utilise les équations suivantes :
\[
\mu(I) = \mu_s + \frac{\mu_2 - \mu_s}{1 + \frac{I_0}{I}}
\]
\[
\Phi(I) = \Phi^{\max} - bI
\]

Les coefficients $\mu_s,\ \mu_2,\ I_0,\ \Phi_{\max},\ b$ sont déterminés empiriquement.

\begin{figure}
    \centering
    {\small
        \input{mu_I_fit.tex}
    }
    \caption{$\mu(I)$ ($\varepsilon_{yy} = 70\%$)}
\end{figure}

\begin{figure}
    \centering
    {\small
        \input{Pack_I_fit.tex}
    }
    \caption{$\Phi(I)$ ($\varepsilon_{yy} = 70\%$)}
\end{figure}

\subsection{Deuxième méthode}

La deuxième méthode, proposée dans l'article 
\href{https://link-springer-com.sid2nomade-1.grenet.fr/article/10.1007/s10035-024-01459-7}{Scaling laws for quasi-statically deforming granular soil at critical state [2024] (Fei, Jianbo et al.)}, 
introduit un nombre d'inertie quasi-statique $Q$ qui tient compte du degré de compaction $\Phi_0$ :

\[
Q = \left[ \Phi_0  \ln \left( I \right) + \alpha \right]
\]
où $\alpha = 30$.

\[
\mu = \xi Q + C
\]

Les coefficients $\xi,\ C,\ \Phi_0$ sont déterminés empiriquement ($\xi \approx 0.06$ et $C \approx 0.2$).

\begin{figure}
    \centering
    {\small
        \input{Q_mu.tex}
    }
    \caption{$\mu(Q)$ avec $Q = f(\Phi_0, I)$}
\end{figure}


\bibliographystyle{alpha}
\bibliography{sample}

% Example citation to avoid "I found no \citation commands" error
\nocite{*}


\end{document}


% \subsection{How to add Citations and a References List}

% You can simply upload a \verb|.bib| file containing your BibTeX entries, created with a tool such as JabRef. You can then cite entries from it, like this: \cite{greenwade93}. Just remember to specify a bibliography style, as well as the filename of the \verb|.bib|. You can find a \href{https://www.overleaf.com/help/97-how-to-include-a-bibliography-using-bibtex}{video tutorial here} to learn more about BibTeX.

% If you have an \href{https://www.overleaf.com/user/subscription/plans}{upgraded account}, you can also import your Mendeley or Zotero library directly as a \verb|.bib| file, via the upload menu in the file-tree.
