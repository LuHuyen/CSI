\documentclass[12pt]{report}
\usepackage[utf8]{inputenc}
\usepackage{amsmath}
\usepackage{graphicx}
\usepackage{hyperref}
\usepackage{color}

\begin{document}

\section{Comparaison des équations utilisées}

Expérimental : 
\[
I =  \frac{\dot{\varepsilon} \cdot \bar{a}}{\sqrt{\sigma_{33}/\rho_s}}; \quad \mu = \sin(\varphi)
\]

Simulation : 
\[
I = \dot{\varepsilon} \times \sqrt {\frac{m}{\sigma_{33}\times \bar{a}}}  
= \dot{\varepsilon} \times \sqrt {\frac{\frac{4}{3} \pi \frac{\bar{a}^3}{8} \times \rho_s}{\sigma_{33}\times \bar{a}}} 
= \dot{\varepsilon} \times \sqrt{\frac{\pi}{6}} \sqrt {\frac{\bar{a}^2 \rho_s}{\sigma_{33}}} 
= \boxed{\sqrt{\frac{\pi}{6}}} \times \frac{\dot{\varepsilon} \cdot \bar{a}}{\sqrt{\sigma_{33}/\rho_s}}
\]
\[
\mu = \tan(\varphi)
\]

\section{Rhéologie $\mu(I)$ résiduel}

\subsection{Première méthode}

La première méthode utilise les équations suivantes :
\[
\mu(I) = \mu_s + \frac{\mu_2 - \mu_s}{1 + \frac{I_0}{I}}
\]
\[
\Phi(I) = \Phi^{\max} - bI
\]

Les coefficients $\mu_s,\ \mu_2,\ I_0,\ \Phi_{\max},\ b$ sont déterminés empiriquement.

\begin{figure}
    \centering
    {\small
        \input{mu_I_fit.tex}
    }
    \caption{$\mu(I)$ ($\varepsilon_{yy} = 70\%$)}
\end{figure}

\begin{figure}
    \centering
    {\small
        \input{Pack_I_fit.tex}
    }
    \caption{$\Phi(I)$ ($\varepsilon_{yy} = 70\%$)}
\end{figure}

\subsection{Deuxième méthode}

La deuxième méthode, proposée dans l'article 
\href{https://link-springer-com.sid2nomade-1.grenet.fr/article/10.1007/s10035-024-01459-7}{Scaling laws for quasi-statically deforming granular soil at critical state [2024] (Fei, Jianbo et al.)}, 
introduit un nombre d'inertie quasi-statique $Q$ qui tient compte du degré de compaction $\Phi_0$ :

\[
Q = \left[ \Phi_0  \ln \left( I \right) + \alpha \right]
\]
où $\alpha = 30$.

\[
\mu = \xi Q + C
\]

Les coefficients $\xi,\ C,\ \Phi_0$ sont déterminés empiriquement ($\xi \approx 0.06$ et $C \approx 0.2$).

\begin{figure}
    \centering
    {\small
        \input{Q_mu.tex}
    }
    \caption{$\mu(Q)$ avec $Q = f(\Phi_0, I)$}
\end{figure}

\end{document}